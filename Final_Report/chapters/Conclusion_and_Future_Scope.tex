\chapter{Conclusion and Future Scope}
\section{Conclusion}
This project was something which required in-depth view in analysis of the problems. This work presents a new deep neural network architecture for automated facial expression recognition. The proposed network consists of two convolutional layers each followed by max pooling and then four Inception layers. The Inception layers increase the depth and width of the network while keeping the computational budget constant. The proposed approach is a single component architecture that takes registered facial images as the input and classifies them into either of the six basic expressions or the neutral. 

We evaluated our proposed architecture in both subject independent and cross-database manners on two well-known publicly available databases. The main problem in this project was encountered during the research process, it is the unavailability of we classified and versatile dataset of huge size.

The clear advantage of the proposed method over conventional CNN methods (i.e. shallower or thinner networks) is gaining increased classification accuracy on both the subject independent and cross-database evaluation scenarios while reducing the number of operations required to train the network.

Completing such a task required one’s undivided attention and it was certainly challenging to do it with other academic responsibilities. A great deal of things were learned while working on this project.  The whole experience of working on this project and contributing to a few others has been very rewarding as it has given great opportunities to learn new things and get a firmer grasp on already known technologies. Here is a reiteration of some of the technologies I have encountered, browsed and learned:
\begin{itemize}
	\item Technology Used:
	\begin{itemize}
		\item Python
		\item Tensorflow
		\item TFLearn
		\item Deep Neural Network
		\item Face Detection
		\item Emotion Classification
		\item OpenCV
		\item Matplotlib
	\end{itemize}
	\item Tools Used:
	\begin{itemize}
		\item PyCharm\\PyCharm is an Integrated Development Environment (IDE) used in computer programming, specifically for the Python language.
		\item Git\\Git is a distributed revision control and source code management (SCM) system with an emphasis on speed, data integrity, and support for distributed, non-linear workflows.
	\end{itemize}
\end{itemize}

\section{Future Scope}
Being a Open Source Project the code of this project is freely available to other th work on it. Being an Open Source project there is a constant flow of suggestion and demands by people.

An implementable and robust real time model for these applications can be a scope for future work. Implementation in Multicore CPU or General Purpose Graphics Processing Unit like NVIDIA or AMD GPUs for cloud based platforms can also be done in future. With a lot of future scope, a development work to can be done for a user interface using android or java or C++ for portable devices. 

Artificial Intelligence is a new and booming field and there is a lot to discover.

The general experimental evaluation of the face expressional
system guarantees better face recognition rates. Having examined techniques to cope with expression variation, in future it may be investigated in more depth about the face classification problem and optimal fusion of color and depth information. Further study can be laid down in the direction of allele of gene matching to the geometric factors of the facial expressions. The genetic property evolution framework for facial expressional system can be studied to suit the requirement of different security models such as criminal detection,
governmental confidential security breaches etc.