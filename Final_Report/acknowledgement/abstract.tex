\begin{center}
	\Huge \textbf{Abstract}
\end{center}
A key step in the humanization of robotics is the ability to classify the emotion of the human operator. In this report we present the design of an artificially intelligent system capable of emotion recognition trough facial expressions. Three promising neural network architectures are customized, trained, and subjected to various classification tasks, after which the best performing network is further optimized. The applicability of the final model is portrayed in a live video application that can instantaneously return the emotion of the user.

Automated Facial Expression Recognition(FER) has remained a challenging and interesting problem. Despite efforts made in developing various methods for FER, existing approaches traditionally lack generalizability when applied to unseen images or those that are captured in wild setting. Most of the existing approaches are based on engineered features where the classifier’s hyper parameters are tuned to give best recognition accuracies across a single database, or a small collection of similar databases. Nevertheless, the results are not significant when they are applied to novel data. 

Deep learning based approaches to facial analysis and video analysis have recently demonstrated high performance on a variety of key tasks such as face recognition, emotion recognition and activity recognition. In the case of video, information often must be aggregated across a variable length sequence of frames to produce a classification result. Prior work using convolutional neural networks (CNNs) for emotion recognition in video has relied on temporal averaging and pooling operations reminiscent of widely used approaches for the spatial aggregation of information